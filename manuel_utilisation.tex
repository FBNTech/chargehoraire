\documentclass[12pt,a4paper]{article}
\usepackage[utf8]{inputenc}
\usepackage[french]{babel}
\usepackage{graphicx}
\usepackage{hyperref}
\usepackage{xcolor}
\usepackage{geometry}
\usepackage{fancyhdr}
\usepackage{tcolorbox}
\usepackage{enumitem}

\geometry{margin=2.5cm}
\pagestyle{fancy}
\fancyhf{}
\fancyhead[L]{Manuel d'utilisation ACH}
\fancyhead[R]{\thepage}
\fancyfoot[C]{Attribution des Charges Horaires}

\hypersetup{
    colorlinks=true,
    linkcolor=blue,
    urlcolor=cyan,
    pdftitle={Manuel d'utilisation ACH},
}

\begin{document}

% Page de garde personnalisée
\begin{titlepage}
    \centering
    
    % Logo
    \vspace*{2cm}
    \includegraphics[width=0.4\textwidth]{static/images/ACHLogo.png}
    
    \vspace{2cm}
    
    % Titre principal
    {\Huge \textbf{ACH}}
    
    \vspace{0.5cm}
    
    {\LARGE Attribution des Charges Horaires}
    
    \vspace{2cm}
    
    % Sous-titre
    {\Large \textbf{Manuel d'Utilisation}}
    
    \vspace{1cm}
    
    {\large Guide pratique pour la gestion des attributions, \\
    la création d'horaires et le suivi des enseignements}
    
    \vfill
    
    % Informations en bas de page
    \begin{tcolorbox}[colback=blue!5!white,colframe=blue!75!black,width=0.8\textwidth]
        \centering
        \textbf{FBNTech} \\
        \vspace{0.3cm}
        Version 1.0 \\
        \today
    \end{tcolorbox}
    
    \vspace{1cm}
    
\end{titlepage}

% Table des matières
\tableofcontents
\newpage

\section{Introduction}

ACH (Attribution des Charges Horaires) est un système de gestion permettant d'attribuer les cours aux enseignants, de créer des horaires et de suivre la progression des enseignements.

\subsection{Connexion au système}

\begin{enumerate}
    \item Accédez à l'URL du système
    \item Entrez votre nom d'utilisateur et mot de passe
    \item Cliquez sur \textbf{Connexion}
\end{enumerate}

Après connexion, vous arrivez sur la page d'accueil affichant l'année académique en cours.

\subsection{Navigation}

Le menu latéral gauche (sidebar) contient les icônes principales. Passez la souris sur chaque icône pour voir sa description.

\section{Gestion des Attributions}

Une attribution consiste à assigner un enseignant à un ou plusieurs cours avec un volume horaire défini.

\subsection{Accéder aux attributions}

\begin{enumerate}
    \item Cliquez sur l'icône \textbf{Attributions} (icône de tâches) dans la sidebar
    \item Vous accédez à la liste de toutes les attributions existantes
\end{enumerate}

\subsection{Créer une nouvelle attribution}

\subsubsection{Étape 1 : Ouvrir le formulaire}

\begin{enumerate}
    \item Sur la page des attributions, cliquez sur le bouton \textbf{Nouvelle attribution} (bouton vert en haut à droite)
\end{enumerate}

\subsubsection{Étape 2 : Remplir le formulaire}

\begin{tcolorbox}[colback=blue!5!white,colframe=blue!75!black,title=Champs du formulaire]
\begin{itemize}[leftmargin=*]
    \item \textbf{Enseignant} : Sélectionnez l'enseignant dans la liste déroulante
    \item \textbf{Code UE} : Sélectionnez le cours à attribuer
    \item \textbf{Type d'heures} : Choisissez parmi :
    \begin{itemize}
        \item CMI (Cours Magistral Intégré)
        \item TD (Travaux Dirigés)
        \item TP (Travaux Pratiques)
    \end{itemize}
    \item \textbf{Nombre d'heures} : Indiquez le volume horaire attribué
\end{itemize}
\end{tcolorbox}

\subsubsection{Étape 3 : Enregistrer}

\begin{enumerate}
    \item Vérifiez les informations saisies
    \item Cliquez sur \textbf{Enregistrer}
    \item Un message de confirmation s'affiche
\end{enumerate}

\begin{tcolorbox}[colback=yellow!5!white,colframe=yellow!75!black,title=Important]
Le système vérifie automatiquement que le nombre d'heures attribuées ne dépasse pas le volume horaire total du cours.
\end{tcolorbox}

\subsection{Consulter les attributions}

\subsubsection{Vue par liste}

Dans la page \textbf{Attributions} :
\begin{itemize}
    \item Chaque ligne affiche : Enseignant, Cours, Type d'heures, Volume horaire
    \item Utilisez la barre de recherche pour filtrer les résultats
    \item Cliquez sur l'icône \textbf{Modifier} pour changer une attribution
    \item Cliquez sur l'icône \textbf{Supprimer} pour retirer une attribution
\end{itemize}

\subsubsection{Vue par charges}

Cliquez sur \textbf{Liste des charges} dans la sidebar pour voir :
\begin{itemize}
    \item Le récapitulatif par enseignant
    \item Le volume horaire total de chaque enseignant
    \item La répartition par type d'heures (CMI, TD, TP)
\end{itemize}

\textbf{Filtres disponibles :}
\begin{itemize}
    \item Par enseignant
    \item Par département
    \item Par semestre
    \item Par année académique
\end{itemize}

\subsection{Modifier une attribution}

\begin{enumerate}
    \item Dans la liste des attributions, cliquez sur l'icône \textbf{Modifier} (crayon)
    \item Modifiez les champs nécessaires
    \item Cliquez sur \textbf{Enregistrer}
\end{enumerate}

\subsection{Supprimer une attribution}

\begin{enumerate}
    \item Dans la liste, cliquez sur l'icône \textbf{Supprimer} (poubelle)
    \item Confirmez la suppression dans la boîte de dialogue
\end{enumerate}

\section{Création des Horaires}

Le générateur d'horaires permet de créer des emplois du temps hebdomadaires pour chaque classe.

\subsection{Accéder au générateur d'horaires}

\begin{enumerate}
    \item Cliquez sur l'icône \textbf{Horaires} (calendrier) dans la sidebar
    \item Vous accédez au générateur d'horaires
\end{enumerate}

\subsection{Créer un horaire}

\subsubsection{Étape 1 : Sélectionner les paramètres}

\begin{tcolorbox}[colback=green!5!white,colframe=green!75!black,title=Paramètres de base]
\begin{enumerate}[leftmargin=*]
    \item \textbf{Année académique} : Sélectionnez l'année en cours (ex: 2024-2025)
    \item \textbf{Semaine} : Choisissez la semaine concernée (ex: Semaine 1)
    \item \textbf{Classe} : Sélectionnez la classe (ex: L3 Informatique)
\end{enumerate}
\end{tcolorbox}

\subsubsection{Étape 2 : Ajouter des créneaux horaires}

Pour chaque cours de la semaine :

\begin{enumerate}
    \item \textbf{Sélectionnez le cours (UE)} : Choisissez dans la liste déroulante
    \item \textbf{Sélectionnez l'enseignant} : Choisissez l'enseignant assigné
    \item \textbf{Définissez le créneau} :
    \begin{itemize}
        \item \textbf{Jour} : Lundi, Mardi, Mercredi, Jeudi, Vendredi ou Samedi
        \item \textbf{Heure de début} : Format HH:MM (ex: 08:00)
        \item \textbf{Heure de fin} : Format HH:MM (ex: 10:00)
        \item \textbf{Salle} : Numéro ou nom de la salle (ex: Amphi A, Salle 101)
    \end{itemize}
    \item Cliquez sur \textbf{Ajouter à l'horaire}
    \item Le créneau apparaît dans la grille horaire
\end{enumerate}

\begin{tcolorbox}[colback=red!5!white,colframe=red!75!black,title=Vérification automatique]
Le système détecte automatiquement les conflits :
\begin{itemize}
    \item Enseignant déjà occupé au même moment
    \item Salle déjà réservée
    \item Chevauchement d'horaires
\end{itemize}
Un message d'erreur s'affiche en cas de conflit.
\end{tcolorbox}

\subsubsection{Étape 3 : Visualiser l'horaire}

Au fur et à mesure de l'ajout des créneaux :
\begin{itemize}
    \item La grille horaire se remplit automatiquement
    \item Chaque créneau affiche : Cours, Enseignant, Salle
    \item Les jours sont en colonnes (Lundi à Samedi)
    \item Les heures sont en lignes
\end{itemize}

\subsubsection{Étape 4 : Générer le PDF}

\begin{enumerate}
    \item Une fois tous les créneaux ajoutés, cliquez sur \textbf{Générer l'horaire PDF}
    \item Le système génère un document PDF avec :
    \begin{itemize}
        \item En-tête : Nom de la classe, Semaine, Année académique
        \item Grille horaire complète
        \item Informations des cours, enseignants et salles
    \end{itemize}
    \item Le PDF s'ouvre automatiquement
    \item Vous pouvez l'imprimer ou le télécharger
\end{enumerate}

\subsection{Modifier un créneau}

\begin{enumerate}
    \item Dans la grille horaire, cliquez sur le créneau à modifier
    \item Modifiez les informations (heure, salle, etc.)
    \item Cliquez sur \textbf{Mettre à jour}
\end{enumerate}

\subsection{Supprimer un créneau}

\begin{enumerate}
    \item Cliquez sur l'icône \textbf{Supprimer} (poubelle) du créneau
    \item Confirmez la suppression
    \item Le créneau disparaît de la grille
\end{enumerate}

\section{Suivi des Enseignements}

Le module de suivi permet d'enregistrer et de consulter la progression des cours.

\subsection{Accéder au tableau de bord de suivi}

\begin{enumerate}
    \item Cliquez sur l'icône \textbf{Suivi} (graphique) dans la sidebar
    \item Vous accédez au tableau de bord de suivi
\end{enumerate}

\subsection{Consulter le tableau de bord}

Le tableau de bord affiche :

\subsubsection{Statistiques globales}

\begin{itemize}
    \item \textbf{Semaine actuelle} : Semaine en cours
    \item \textbf{Progression globale} : Pourcentage d'avancement général
    \item \textbf{Heures réalisées / Heures allouées} : Barre de progression
\end{itemize}

\subsubsection{Tableau de suivi des cours}

Pour chaque cours :
\begin{itemize}
    \item Code UE
    \item Intitulé du cours
    \item Classe
    \item Heures réalisées
    \item Progression en pourcentage (avec barre de progression colorée)
\end{itemize}

\textbf{Code couleur :}
\begin{itemize}
    \item \textcolor{blue}{Bleu} : Moins de 75\% de progression
    \item \textcolor{orange}{Orange} : Entre 75\% et 99\%
    \item \textcolor{green}{Vert} : 100\% ou plus (cours terminé)
\end{itemize}

\subsubsection{Tableau de suivi des enseignants}

Pour chaque enseignant :
\begin{itemize}
    \item Nom et prénom
    \item Heures effectuées
    \item Heures allouées
    \item Taux de réalisation
\end{itemize}

\subsection{Ajouter une progression}

\subsubsection{Étape 1 : Accéder à la gestion}

\begin{enumerate}
    \item Sur le tableau de bord, cliquez sur \textbf{Gestion de suivi des enseignements}
    \item Vous accédez à la liste des progressions enregistrées
\end{enumerate}

\subsubsection{Étape 2 : Créer une nouvelle progression}

\begin{enumerate}
    \item Cliquez sur \textbf{Ajouter une progression} (bouton vert)
    \item Remplissez le formulaire :
\end{enumerate}

\begin{tcolorbox}[colback=blue!5!white,colframe=blue!75!black,title=Champs du formulaire]
\begin{itemize}[leftmargin=*]
    \item \textbf{Cours (UE)} : Sélectionnez le cours concerné
    \item \textbf{Enseignant} : Sélectionnez l'enseignant
    \item \textbf{Semaine} : Choisissez la semaine
    \item \textbf{Heures effectuées} : Nombre d'heures réalisées cette semaine
    \item \textbf{Date} : Date de la séance (optionnel)
    \item \textbf{Observations} : Commentaires ou remarques (optionnel)
\end{itemize}
\end{tcolorbox}

\subsubsection{Étape 3 : Enregistrer}

\begin{enumerate}
    \item Vérifiez les informations
    \item Cliquez sur \textbf{Enregistrer}
    \item La progression est ajoutée au système
    \item Les statistiques du tableau de bord sont mises à jour automatiquement
\end{enumerate}

\subsection{Modifier une progression}

\begin{enumerate}
    \item Dans la liste des progressions, cliquez sur l'icône \textbf{Modifier}
    \item Modifiez les champs nécessaires
    \item Cliquez sur \textbf{Enregistrer}
\end{enumerate}

\subsection{Imprimer le rapport de suivi}

\begin{enumerate}
    \item Sur le tableau de bord, cliquez sur \textbf{Imprimer le rapport}
    \item Le système génère un PDF avec :
    \begin{itemize}
        \item Statistiques globales
        \item Tableaux détaillés des cours
        \item Tableaux détaillés des enseignants
        \item Graphiques de progression
    \end{itemize}
    \item Le PDF s'ouvre automatiquement pour impression ou téléchargement
\end{enumerate}

\subsection{Filtrer les données}

Sur le tableau de bord, utilisez les filtres pour :
\begin{itemize}
    \item Rechercher un cours spécifique (par code ou intitulé)
    \item Filtrer par semestre (impairs, pairs, ou tous)
    \item Rechercher un enseignant
    \item Filtrer par classe
\end{itemize}

\section{Conseils et bonnes pratiques}

\subsection{Pour les attributions}

\begin{itemize}
    \item Vérifiez que l'enseignant a la spécialité correspondant au cours
    \item Assurez-vous de ne pas dépasser le volume horaire du cours
    \item Consultez régulièrement la liste des charges pour équilibrer la répartition
\end{itemize}

\subsection{Pour les horaires}

\begin{itemize}
    \item Commencez par les cours à horaires fixes
    \item Vérifiez la disponibilité des salles avant d'ajouter un créneau
    \item Laissez des pauses entre les cours (minimum 15 minutes)
    \item Évitez de programmer des cours après 18h00 si possible
    \item Générez le PDF régulièrement pour vérifier la cohérence
\end{itemize}

\subsection{Pour le suivi}

\begin{itemize}
    \item Enregistrez les progressions chaque semaine
    \item Ajoutez des observations pour les cours en retard
    \item Consultez régulièrement le tableau de bord pour identifier les retards
    \item Imprimez le rapport mensuel pour les réunions pédagogiques
\end{itemize}

\section{Résolution de problèmes}

\subsection{Erreur lors de l'attribution}

\textbf{Problème :} "Le volume horaire dépasse le total du cours"

\textbf{Solution :} Vérifiez le volume horaire total du cours et les attributions déjà effectuées. Ajustez le nombre d'heures en conséquence.

\subsection{Conflit d'horaires}

\textbf{Problème :} "L'enseignant est déjà occupé à cette heure"

\textbf{Solution :} Choisissez un autre créneau horaire ou un autre enseignant pour ce cours.

\subsection{Progression non visible}

\textbf{Problème :} La progression ajoutée n'apparaît pas dans le tableau de bord

\textbf{Solution :} Rafraîchissez la page (F5) ou vérifiez que vous avez bien sélectionné la bonne semaine et le bon cours.

\section{Contact et support}

Pour toute question ou problème technique, contactez l'administrateur système.

\vspace{1cm}

\begin{center}
\textit{Fin du manuel d'utilisation}
\end{center}

\end{document}
